\documentclass{beamer}
% Language
% Use the option "language=EN" to set the beamer theme in English. Use the option "language=ES" to set the beamer theme in Spanish.

% Colors
% Use the option "color=white" to set the background in white and the bottom bar in blue. User the option "color=blue" to set the background in blue and the bottom bar in white. Use the option "color=blue2" to set the background in blue and the bottom bar in blue.

\usetheme[language=ES, color=white]{EAFIT}


%Paquetes Necesarios
\usepackage{helvet}
\usepackage{tikz}
\usepackage{ifthen}
\usepackage[utf8]{inputenc}
\usepackage[spanish]{babel}
\selectlanguage{spanish}
\usepackage{ragged2e}
\justifying
\usepackage{xcolor}
\usepackage{graphicx}
% Configura tipo de letra por Helvet
\renewcommand{\familydefault}{\sfdefault}


\author{Alejandro Gómez Montoya}
\title{Plantilla Beamer Theme EAFIT No Oficial}
\date{\today}
% Se define la Escuela
\def\escuela{Escuela de Ciencias}
% Se define el Departamento
\def\departamento{Departamento de Ciencias F\'isica}
% Se define la Carrera
\def\carrera{Ingenier\'ia F\'isica}





\begin{document}
	%Portada Inspira Crea Transforma
	\begin{frame}
		
	\end{frame}
	%Titulo
	\begin{frame}
		\titlepage
	\end{frame}
	%Tabla de Contenidos
	\begin{frame}{Contenido}
		\tableofcontents
	\end{frame}
	\section{Neque porro}
	\begin{frame}{Neque porro}
		Lorem ipsum dolor sit amet, consectetur adipiscing elit. Vivamus ac aliquam metus, sed malesuada libero. Morbi pharetra porttitor lectus non auctor. Praesent congue elit eget aliquam faucibus. Donec at cursus lectus. Donec bibendum at odio viverra condimentum.	
	\end{frame}
	\section{quisquam est qui }
	\begin{frame}{quisquam est qui }
		Phasellus non varius dolor, vitae maximus turpis. Fusce mollis a libero et pharetra. Nunc urna nibh, lacinia nec tellus eleifend, feugiat aliquam arcu. Pellentesque posuere elit et nunc tincidunt, eu egestas purus sollicitudin. In feugiat quam elit, at viverra libero congue non.
	\end{frame}
\end{document}
